\documentclass[10pt,a4paper]{article}
\usepackage[utf8]{inputenc}
\usepackage[catalan]{babel}
\usepackage[margin=2.5cm]{geometry}
\usepackage[procnames]{listings}
\usepackage{graphicx}
\usepackage{titling}
\usepackage{fancyhdr}
\usepackage{float}
\usepackage{amsmath}
\usepackage{enumitem}
\usepackage{hyperref}
\graphicspath{{images/}}
\usepackage[x11names]{xcolor}
\usepackage{listings}
\usepackage{longtable}
 
\title{XtiSpd: Comunicació amb llum visible}
\author{Sudan Wu}
\date{\today}


\pagestyle{fancy}
\fancyhf[lh]{XtiSpd: Treball}
\fancyhf[rh]{\theauthor}
\fancyhf[cf]{\thepage}

\begin{document}

\begin{titlepage}
    \begin{center}
        \includegraphics[height=1.8cm]{logoUdG}\\\vfill
    \end{center}
    \center
    {\huge \bfseries Xarxes troncals i Serveis públics de dades}\\[0.5cm]
    {\Huge \bfseries Visible light communication} \\[0.5cm]
    {\Huge \bfseries (VLC)} \\[0.5cm]

    \vfill
    \begin{center} \large
        {Sudan Wu, sudan88792@gmail.com} \\[0.25cm]
        {\today}\\ [1cm]
    \end{center}

\end{titlepage}

\tableofcontents

\clearpage

\section{Que és la Comunicació amb llum visible}

\subsection*{Introducció}
\addcontentsline{toc}{subsection}{Introducció}
La comunicació amb llum visible (coneguda com a VLC, acrònim en anglès de "visible light communication") és un mitjà transmissor de dades que utilitza la llum entre 400 i 800 THz (780-375 nm). VLC és un subconjunt de tecnologies de comunicacions òptiques sense fil.



\begin{figure}[h!]
    \centering
    \includegraphics[width=100mm]{llumVisible.png}
    \caption{Espectre del llum visible}
\end{figure}


La tecnologia  VLC utilitza llums fluorescents (làmpades normals, no necessita dispositius especials) per transmetre senyals a una velocitat de 10 kbit/s, o LEDs que pot assolir velocitats de fins a 500 Mbit/s. RONJA (Reasonable Optical Near Joint Access) aconsegueix assolir una velocitat d'Ethernet completa (10 Mbit/s) sobre la mateixa distància gràcies a una òptica més gran i LED més potents.

La VLC pot ser utilitzada com un mitjà transmissor de computació ubiqua, atès que els dispositius que produeixen llum (làmpades d'interior exterior, televisors, senyals de trànsit, lluminosos comercials, fars de vehicles3) utilitzats a tot arreu, es poden aprofitar. El fer ús de la llum visible és també menys perillosa per a aplicacions que utilitzen una gran potència, ja que, els éssers humans poden percebre i protegirse-se del dany que pugui ocasionar-los.



\section*{Tecnologies inalàmbriques}
\addcontentsline{toc}{subsection}{Tecnologies inalàmbriques}

Els subconjunts de tecnologies de comunicació òptiques sense fil són:
\begin{itemize}
    \item Li-Fi: és una nova tecnologia de comunicació sense fils per proporcionar accés a Internet a través de la llum.
    \item Infraroig (IR): Encara que no és visible per a l'ull humà, la llum infraroja s'utilitza en diverses tecnologies sense fil, com ara els controls remots, els dispositius de comunicació a curta distància i les transmissions de dades sense fil en alguns contextos.
    \item Comunicació Òptica d'Espai Lliure (FSO): Aquesta tecnologia utilitza feixos de llum làser per a la transmissió de dades sense cables a través de l'espai lliure, com ara entre dos edificis. És especialment útil en línies de vista directa sense obstruccions.
    \item Comunicació Òptica Inalàmbrica (OWC): Aquesta tecnologia inclou diverses formes de transmissió de dades sense fil a través de llum òptica, incloent l'ús de làsers i LED.
    \item Comunicació Òptica en Entorns Controlats: En alguns entorns, com ara entorns d'oficines o fàbriques, es poden utilitzar sistemes d'òptica sense fil per a la transmissió de dades en lloc de les tecnologies tradicionals sense fil.
\end{itemize}

Cada tecnologia té les seves pròpies aplicacions, avantatges i desavantatges. La selecció de la tecnologia òptica sense fil adequada depèn dels requisits específics de l'aplicació i les condicions de l'entorn.

L'estàndard IEEE 802.15.7 estableix les especificacions per a la comunicació sense fils de curt abast utilitzant tecnologies òptiques, com LED i làser. Defineix els protocols i especificacions de capa física i denllaç de dades per a les comunicacions a través de la llum visible.

\section{Historia de les comunicacions inalámbriques}
\section{Què és LiFi?}

\subsection*{Què és la tecnologia Li-Fi i com funciona?}
\addcontentsline{toc}{subsection}{Què és la tecnologia Li-Fi i com funciona?}

Li-Fi és una forma de comunicació òptica que utilitza la llum visible per a la transmissió de dades. La tecnologia Li-Fi, que significa "Light Fidelity", és un sistema de comunicació sense fil que utilitza llum LED (diodos emissors de llum) per a transmetre dades a una velocitat molt alta. Aquesta tecnologia funciona modulant la intensitat de la llum a una taxa molt ràpida, invisible per a l'ull humà, per transmetre informació.

La tecnologia Li-Fi suposa una gran millora en comparació amb el Wi-Fi a tots els nivells. Per començar, la velocitat de transmissió és fins a 100 vegades més ràpida!

\begin{figure}[h!]
    \centering
    \includegraphics[width=80mm]{lifi.png}
    \caption{visió general de mercat VLC}
    \label{fig:method}
\end{figure}


\subsection*{Sostenibilitat: menor cost i més eficiència}
\addcontentsline{toc}{subsection}{Sostenibilitat: menor cost i més eficiència}

Els seus avantatges no estan només a la velocitat. S'estima que en un futur proper podrem transmetre dades a través de l'energia solar, cosa que facilitarà l'accés a persones sense Internet i amb recursos d'electricitat limitats. El funcionament de la tecnologia Li-Fi estalviarà costos a les llars i, sobretot, als llocs de treball. Podria funcionar sense dispositius electrònics com ara rúters, mòdems, repetidors, amplificadors d'ona i antenes.

Aquests dispositius, que actualment estan connectats a la xarxa elèctrica les 24 hores del dia, els 7 dies de la setmana, deixarien de consumir electricitat i la seva funció seria reemplaçada per una bombeta LED, que en la majoria dels casos ja està encesa durant el horari de treball, per tant, no significaria un cost extra.

La principal diferència entre Li-Fi i altres formes de comunicació òptica, com ara la fibra òptica, és que Li-Fi no utilitza fils físics per a la transmissió de dades. En canvi, transmet les dades mitjançant la il·luminació LED ja existent en un entorn. Aquesta tecnologia pot oferir velocitats de transmissió de dades molt altes i també pot ser utilitzada en entorns on les comunicacions per ones de ràdio poden ser limitades, com ara en hospitals o en aules on l'ús de ones de ràdio pot estar restringit.

\subsection*{Seguridad contra ataques}
\addcontentsline{toc}{subsection}{Seguridad contra ataques}


Al ser necesario estar en contacto directo con el emisor de haz de luz LED, se refuerza la seguridad informática. Solo los dispositivos iluminados por la misma bombilla pueden interconectarse entre sí, eliminando ataques o intentos de entrada no autorizados desde dispositivos fuera de nuestro espectro de luz.

Una solución brillante, más segura, más rápida y eficiente para optimizar nuestra conexión con el mundo.


\subsection*{Incovenients de Li-Fi}
\addcontentsline{toc}{subsection}{Incovenients de Li-Fi}

Encara que la tecnologia Li-Fi ofereix algunes avantatges, també té alguns inconvenients i limitacions que poden afectar la seva adopció i implementació:

\begin{enumerate}
    \item Dependència de la llum visible: La transmissió de dades a través de Li-Fi depèn de la llum visible, de manera que la comunicació pot veure afectada per la falta de llum, com en entorns foscos o quan les llums estan apagades.
    \item Limitacions d'abast: La cobertura de Li-Fi és generalment més limitada que la del Wi-Fi, ja que la llum no penetra a través de parets o altres obstacles com ho fa la radiació de radiofreqüència.
    \item Interferència de la llum solar: La llum solar pot interferir amb la transmissió de dades Li-Fi, ja que les dues fonts de llum poden ser captades pel receptor, provocant interferències i afectant el rendiment de la comunicació.
    \item Limitacions de mobilitat: La tecnologia Li-Fi pot ser menys adequada per a dispositius en moviment, ja que requereix una connexió visual constant amb la font de llum per a una transmissió de dades eficaç.
\end{enumerate}



\subsection*{Productes Li-Fi}
\addcontentsline{toc}{subsection}{Productes Li-Fi}
\section{Problemes de les cominicacions RF}
\section{Actuals àrees d'investigació}
\section{Productes LiFi}
\section{A cara del futur}
\section{Relacionar IEEE amb VLC}
\section{Altres tecnologies inalàmbriques}
\section{Conclusió}
\section{Bibliografies}

 \url{https://en.wikipedia.org/wiki/Visible_light_communication}
 \\
 \url{https://pandorafms.com/blog/es/tecnologia-lifi/}
 \\
 \url{https://www.maximizemarketresearch.com/market-report/global-visible-light-communication-market/54383/}
 \\
 \url{https://openaccess.uoc.edu/bitstream/10609/95748/7/rarellanoTFG0619memoria.pdf}
 \\
\url{https://gyemo.com/blog/post/que-es-la-tecnologia-li-fi-y-como-funciona.html}
\\
\url{https://lifi.co/es/productos-lifi/}
\\
\url{https://ca.wikipedia.org/wiki/Reflexi%C3%B3_total}
\end{document}