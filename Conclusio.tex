


\subsection*{Què tan a prop estem de fer servir LiFi?}
\addcontentsline{toc}{subsection}{Què tan a prop estem de fer servir LiFi?}
S'ha pronosticat que el LiFi es llançarà al públic en general al començament del 2022.

Una botiga de queviures a França actualment fa servir LiFi per rastrejar les ubicacions dels seus clients a tota la botiga i després poder oferir cupons i altres incentius. Les grans companyies mòbils, com Apple, també comencen a suggerir que els seus futurs dispositius seran compatibles amb LiFi.

pureLiFi està treballant actualment amb socis a moltes indústries, incloses defensa, atenció mèdica, il·luminació, infraestructura de TI, empreses de telecomunicacions i integradors de dispositius. L'empresa també està dedicant molts recursos a la investigació per al desenvolupament i comercialització de productes.



\subsection*{Futur del Li-Fi}
\addcontentsline{toc}{subsection}{Futur del Li-Fi}


La tecnologia LiFi està sent desenvolupada actualment per diferents organitzacions a tot el món.

En aquest moment, el LiFi no pot reemplaçar completament el WiFi com a font de connectivitat, però, hi ha diverses companyies de LiFi que estan treballant àrduament per desenvolupar productes LiFi i comercialitzar el LiFi com la principal tecnologia sense fils. La demanda daccés ràpid a Internet augmenta cada dia i Light Fidelity (LiFi) podria ser la tecnologia per satisfer aquesta demanda.

També en un futur proper, s'han estimat que podràn transmetre dades a través de l'energia solar. Això significa que les persones sense accés a Internet o amb recursos d'electricitat limitats ara es podran connectar a la web sense fils.

El LiFi es considera el futur dInternet. Aquest futur es veu brillant i tots estem desitjant que arribi.
