
Li-Fi és una forma de comunicació òptica que utilitza la llum visible per a la transmissió de dades. La tecnologia Li-Fi, que significa "Light Fidelity", és un sistema de comunicació sense fil que utilitza llum LED (diodos emissors de llum) per a transmetre dades a una velocitat molt alta. Aquesta tecnologia funciona modulant la intensitat de la llum a una taxa molt ràpida, invisible per a l'ull humà, per transmetre informació.


La principal diferència entre Li-Fi i altres formes de comunicació òptica, com ara la fibra òptica, és que Li-Fi no utilitza fils físics per a la transmissió de dades. En canvi, transmet les dades mitjançant la il·luminació LED ja existent en un entorn. Aquesta tecnologia pot oferir velocitats de transmissió de dades molt altes i també pot ser utilitzada en entorns on les comunicacions per ones de ràdio poden ser limitades, com ara en hospitals o en aules on l'ús de ones de ràdio pot estar restringit.




Avantatges: VLC proporciona una comunicació segura com a transmissió de dades perquè no es pot interrompre, cosa que és un problema en la comunicació de radiofreqüència. VLC és un sistema de comunicació basat en la llum, no es veu afectat per les radiacions electromagnètiques dels sistemes de radiofreqüència.


\subsection*{Teoria física que es basa}

\subsection*{Material que utilitza}